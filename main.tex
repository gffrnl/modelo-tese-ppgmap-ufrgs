% %%%%%%%%%%%%%%%%%%%%%%%%%%%%%%%%%%%%%%%%%%%%%%%%%%%%%%%%%%%%%%%%%%%%%%%%%%%%%%%% %
%
%	MODELO LaTeX PARA ELABORACAO DE TESES E DISSERTACOES 
%  ----------------------------------------------------
%
%	BASEADO NA CLASSE PPGMAp.cls (Classe base: report.cls)
%  ------------------------------------------------------
%
%
%	Creditos
%
%		Autor: Guilherme F. Fornel (guilherme.fornel@ufrgs.br)
%	
%		! Se encontrar bugs envie um e-mail relatando !
%
%
%	Pacotes nativos da classe PPGMAp.cls (nao precisam ser adicionados pelo usuario)
%		
%		inputenc (utf8 encoding)
%		fontenc  (T1 encoding)
%		babel    (default: brazil - english, portuguese, french, spanish,
%				    german, russian)
%		etoolbox
%		nomencl  (intoc option)
%		lipsum
%		atbegshi
%		calc
%
%
%	Sobre a compilacao
%
%		E recomendavel a utilizacao da biblioteca/compilador TeX Live 2016 FULL
%		(este modelo foi compilado para teste com a versao 2016.20170123-5)
%
%		Para a instalacao do TeX Live em so's derivados do linux Ubuntu utilizar
%
%			sudo apt-get install texlive-full 
%
%		# Compilando o modelo via terminal do linux:
%
%			Para a compilacao via terminal utilizar as seguintes linhas de 
%			comando:
%
%			pdflatex nome_do_arquivo_principal
%			bibtex nome_do_arquivo_principal   <opcional>
%			makeindex -s nomencl.ist -t nome_do_arquivo_principal.nlg -o nome_do_arquivo_principal.nls nome_do_arquivo_principal.nlo <para lista de simbolos>
%			pdflatex nome_do_arquivo_principal
%			pdflatex nome_do_arquivo_principal
%
%			No caso deste modelo:
%
%			pdflatex main
%			bibtex main   <opcional>
%			makeindex -s nomencl.ist -t main.nlg -o main.nls main.nlo <para lista de simbolos>
%			pdflatex main
%			pdflatex main		
%
%		# Utilizando a ide TeXstudio:
%
%			Usuarios do TeXstudio devem configurar a ide para gerar a lista de 
%			simbolos acessando
%		
% 			Opções -> Configurar TeXstudio -> Compilação -> Comandos do usuário
%					-> Adicionar
%
%			Deve ser adicionado o comando Make Nomenclature com
%
%			Nome do comando:  user0:Make Nomenclature
%			Linha de comando: makeindex -s nomencl.ist -t %.nlg -o %.nls %.nlo
%
%			Para a compilacao utilizar:
%
%			Ferramentas -> Comandos -> PDFLaTeX
%			Ferramentas -> Comandos -> BibteX              <opcional>
%			Ferramentas -> Usuario  -> Make Nomenclature   <para lista de simbolos>
%			Ferramentas -> Comandos -> PDFLaTeX
%			Ferramentas -> Comandos -> PDFLaTeX
%


% %%%%%%%%%%%%%%%%%%%%%%%%%%%%%%%%%%%%%%%%%%%%%%%%%%%%%%%%%%%%%%%%%%%%%%%%%%%%%%%% %
%
%	Opcoes da classe PPGMAp.cls
%
%		Por padrao a classe esta configurada para tese de doutorado com texto em 
%		portugues, folha A4, tamanho de fonte 12pt e configuracao de impressao em
%		ambos os lados da folha. Ao usuario, porem, estao disponiveis algumas opcoes 
%		de configuracao:
%
%		qualification : opcao para texto de qualificacao
%
%		masters : dissertacao de mestrado
%
%		english : texto em ingles
%
%		oneside : opcao para configuracao de impressao em apenas um lado da folha
%					 (ficha catalografica tambem)
%
%		print : opcao de versao final do texto para impressao em apenas um lado 
%				  da folha, com excessao da primeira pagina (folha de rosto e ficha
%				  catalografica). Para tal sao inseridas paginas em branco no 
%				  documento, e o usuario deve selecionar no aplicativo de
%				  impressao do sistema operacional a opcao IMPRIMIR EM AMBOS OS LADOS
%		
%		Demais opcoes da classe base report.cls NAO sao repassadas a classe PPGMAp.cls
%	

% \documentclass[print,masters]{./config/PPGMAp}
\documentclass[masters]{./config/PPGMAp}


% %%%%%%%%%%%%%%%%%%%%%%%%%%%%%%%%%%%%%%%%%%%%%%%%%%%%%%%%%%%%%%%%%%%%%%%%%%%%%%%% %
%
%	Informacoes para os elementos pre-textuais
%
%		Observacoes
%
%			Algumas informacoes sao opcionais, como \coadvisor e \memberd
%
%			Por padrao a data do documento e a data de compilacao e a data da defesa
%			e especificada com o comando \date
%
%			Nas infomacoes onde e possivel especificar o genero do individuo,
%		   a seguinte convencao e adotada: 'a','A','f' ou 'F' para feminino,  
%		   caso contrario masculino (neste caso os colchetes [] NAO precisam
%			ser escritos).
%
%		   (A especificacao da opcao 'o' neste modelo se da somente para exibir que
%			nao ha nenhum efeito sobre o genero, o genero masculino sendo adotado. 
%			Obviamente, no caso de texto em ingles a especificacao de genero tambem 
%			nao tem nenhum efeito)
%

%
% Titulo e data da defesa:
%
\title{Larousse Gastronomique}
\date{novembro}{2002}

%
% Linha de pesquisa:
%
\area{Análise Aplicada}

%
% Palavras-chave:
%	{palavra-chave 1}{palavra-chave 2}{palavra-chave 3}{palavra-chave 4}
%
% " Keywords:
%	  {keyword 1}{keyword 2}{keyword 3}{keyword 4} "
%
\keywords{Dissertação}{Tese}{Mestrado}{Doutorado}

%
% Autor:
% 	[genero<opcional>]{nome}{sobrenome}
%
\author[o]{Hannibal}{Lecter}

%
% Orientador (e, se ha, co-orientador):
% 	[genero<opcional>]{tratamento}{nome}{sobrenome}
%
\advisor{Prof. Dr.}{Nome}{Sobrenome}
%\coadvisor[a]{Profa. Dra.}{Nome}{Sobrenome}

%
% Membros da banca:
% 	{tratamento}{nome}{sobrenome}
%
\membera{Profa. Dra.}{Nome}{Sobrenome}{Instituição A}
\memberb{Prof. Dr.}{Nome}{Sobrenome}{Instituição B}
\memberc{Prof. Dr.}{Nome}{Sobrenome}{Instituição C}

%
% Se ha um quarto membro na banca:
%
%\memberd{Prof. Dr.}{Nome}{Sobrenome}{Instituição D}

%
% Coordenador(a) do PPGMAp:
% 	[genero<opcional>]{tratamento}{primeiro nome}{ultimo nome}
%
\PPGMApcoord{Prof. Dr.}{Ezequia}{Sauter}


% %%%%%%%%%%%%%%%%%%%%%%%%%%%%%%%%%%%%%%%%%%%%%%%%%%%%%%%%%%%%%%%%%%%%%%%%%%%%%%%% %
%
%	Pacotes utilizados, opcoes e comandos definidos pelo usuario
%
%
% \usepackage{mypreamble}
% \usepackage{mypackages}
% \usepackage{mycommands}
% ...
%

\usepackage{./config/pack-and-cmd} % preâmbulo em outro arquivo!

% %%%%%%%%%%%%%%%%%%%%%%%%%%%%%%%%%%%%%%%%%%%%%%%%%%%%%%%%%%%%%%%%%%%%%%%%%%%%%%%% %
%
%	Definicoes de nomenclaturas (siglas, simbolos e constantes fisicas)
%
%		A classe PPGMAp.cls utiliza o pacote nomenclature.sty para a criacao da
%		lista de siglas e simbolos. Isto possibilita ao usuario inserir 
%		facilmente Siglas, Simbolos e Constantes Fisicas respectivamente com os
%		comandos \Abbrev, \Symbol e \PConst definidos na classe PPGMAp.cls
%
%		Apesar de neste modelo os simbolos estarem especificados no preambulo
%		estes comandos (a principio) podem ser especificados tambem em quaisquer 
%     lugares dentro do ambiente de documento.	
%
%		Dupla insercao de simbolos (a principio) e desconsiderada.
%
%		O arquivo .nls necessario para a lista e gerado com o comando
%
%		makeindex main.nlo -s nomencl.ist -o main.nls
%

\Abbrev{SBM}{Sociedade Brasileira de Matemática}%
\Abbrev{SBM}{Sociedade Brasileira de Matemática}% dupla insercao, desconsiderada
\Abbrev{CFD}{Computational Fluid Dynamics}%

\Symbol{$\Delta$}{Operador laplaciano}%
\Symbol{$\mu_N$}{Medida empírica}%

\PConst{$k$}{Constante de Boltzmann}%
	{$1,380 648 52\times 10^{-23}\; J\cdot K^{-1}$}%
\PConst{$\hbar$}{Constante reduzida de Planck}%
	{$6,626 070 15\times 10^{-34}\; J\cdot s/2\pi$}%


% %%%%%%%%%%%%%%%%%%%%%%%%%%%%%%%%%%%%%%%%%%%%%%%%%%%%%%%%%%%%%%%%%%%%%%%%%%%%%%%% %
%
%	Ambiente de documento
%
%		Sao declarados, alem de comandos para impressao de elementos obrigatorios
%		no texto, alguns ambientes de insercao (obrigatorios ou nao) nativos
%		da classe report.cls ou definidos na classe PPGMAp.cls.
%
%		Sao indicados dois sistemas de insercao de referencias bibliograficas,
%		o nativo e o com Bibtex.
%
%		A forma adotada para a declaracao dos elementos no ambiente documento visa 
%		dar alguma liberdade de modificacao do texto ao usuario.
%

\begin{document}

% !TeX root = ../main.tex

% defining tikz decoration -dot-:
\tikzset{
    -dot-/.style={
        decoration={
            markings,
            mark=at position #1 with {
                \fill circle (1.5pt);
            }
        },
        postaction={decorate}
    }
}

% defining tikz decoration ->:
\tikzset{
    ->-/.style={
        decoration={
            markings,
            mark=at position #1 with {
                \arrow{>}
            }
        },
        postaction={decorate}
    }
}

%
%	Elementos pre-textuais:
%
\frontmatter % formatacao do estilo dos elementos pre-textuais

%	Impressao da folha de rosto:
	\coversheet

%	Impressao da ficha catalografica:
	\catalogsheet

%	Impressao da folha de aprovacao:
	\approvalsheet

%	Epigrafe:
%		Entre colchetes e especificada a porcentagem da largura da pagina destinada
%		a epigrafe (padrao 0.5)
	\begin{epigraph}[0.7]
		{
			\normalfont\large
			A$\Gamma$E$\Omega$METPHTO$\Sigma$
			MH$\Delta$EI$\Sigma$
			EI$\Sigma$IT$\Omega$
		} \\[-12pt]
		``Que não entre aqui aquele que não sabe Geometria'' \\
		(Inscrição que marcava a entrada da Academia de Platão)
		% Epigrafe "roubada" da tese de Renato Vaz Linn:
		% https://lume.ufrgs.br/handle/10183/163257
	\end{epigraph}

%	Agradecimentos <ambiente>:
	\begin{acknowledgments}
		\lipsum[1]
	\end{acknowledgments}

%	Impressao do sumario:
	\tableofcontents

%	Impressao da lista de figuras:
	\listoffigures

%	Impressao da lista de tabelas:
	\listoftables

%	Impressao da lista de siglas e simbolos:
%		Entre colchetes e especificado um parametro de comprimento do campo de
%		simbolo (padrao 2).
%	\printnomenclature[6]
	\printnomenclature

%	Resumo <ambiente>:
	\begin{resumo}
		\lipsum[2-4]
	\end{resumo}

%	Abstract <ambiente>:
	\begin{abstract}
		\lipsum[5-7]
	\end{abstract}

%
%	Elementos textuais:
%
\mainmatter % formatacao do estilo dos elementos textuais

%
% Chapters
%

% Chapter: Introduction
\chapter{Introdução}
Teste latexmk
\section{Título da primeira seção}
Texto texto texto texto texto texto texto texto texto texto texto texto texto texto texto texto texto texto texto texto texto.

\subsection{Como inserir citações no texto}

De acordo com \cite{bucur2016nonlocal} texto texto texto texto texto texto texto texto texto texto texto texto texto texto texto texto texto texto texto texto texto texto texto texto texto texto texto.

Texto texto texto  texto texto texto texto texto texto texto texto texto texto texto texto texto texto texto texto texto texto \cite{huang2014numerical,warnatz06}.

\section{Figuras e Tabelas}\label{sec3}

Instruções para a introdução de figuras e tabelas.

%\begin{obs}
%Não esqueçam: tabelas e figuras têm legendas.
%\end{obs}

\subsection{Exemplo: como incluir figuras}
OBS: A legenda deve vir abaixo da figura.

Na Figura \ref{figura01} é possível perceber que texto texto texto texto.
\begin{figure}
\centering
\includegraphics[width=.3\textwidth]{chap/logoppgmap.png}
\caption{ {\small Logo do PPGMAp, UFRGS, Porto Alegre.}}
\label{figura01}
\end{figure}

\subsection{Exemplo: como incluir tabelas}

Na confecção das tabelas é recomendado usar o ambiente \texttt{table}, com legenda acima da tabela e as entradas centralizadas nas colunas, como na Tabela \ref{tabela01}.
\begin{table}[ht] \label{tabela01}
\caption{ \  Metodologia 1- Boole}
\begin{center}
\begin{footnotesize}
\begin{tabular}{ccccccc}
\hline
 $x$  & N=1001  & N=2001  & N=4001  & N=8001
\\ \hline   
0.00  &  0.000467  &  0.000467  &  0.000468    &  0.000468    &  \\ 
10.00  &  0.031540  &  0.031546  &  0.031546    &  0.031546    &  \\
20.00  &  0.612111  &  0.612161  &  0.612163    &  0.612163    & \\ 
30.00  &  12.320687  &  12.318744  &  12.318006    &  12.317711    &  \\ 
40.00  &  1.705255  &  1.705382  &  1.705394    &  1.705398    & \\ 
50.00  &  1.506538  &  1.507221  &  1.507602    &  1.507797    & \\ 
60.00  &  21.900884  &  21.894519  &  21.892193    &  21.891369    &\\ 
70.00  &  1.072764  &  1.072963  &  1.073030    &  1.073067    & \\ 
80.00  &  0.055370  &  0.055387  &  0.055391    &  0.055393    & \\ 
90.00  &  0.002853  &  0.002854  &  0.002854    &  0.002855    & \\ 
100.00  &  0.000042  &  0.000042  &  0.000042    &  0.000042    & \\ 
  \hline
\end{tabular}
\end{footnotesize}
\end{center}
\label{table6P}
\end{table}



\subsection{Exemplo: como incluir equações}

As equações mais importantes são numeradas sequencialmente no texto, com a numeração automaticamente colocada à direita usando o comando \verb!\label{nome-da-equacao}! para identificá-las. A chamada \verb!\eqref{nome-da-equacao}! faz referência à equação, no texto.

EXEMPLO: Considere o sistema de equações \eqref{equm}, dado por:
%\begin{equation}\label{equm}
%\frac{\partial u}{\partial t}=\frac{\partial u}{\partial x}, \quad  %\mathrm{onde} \; u=u(x,t)
%\end{equation}
\begin{eqnarray} \label{equm}
\left\{
\begin{array}{ll}
\vspace{0.1cm}
\displaystyle \dot{H} &= r\left(1-\frac{H}{K}\right) H-n_{1}H-\beta PH, \\
\vspace{0.1cm}
\displaystyle \dot{I} &= \beta HP - I\left(m_{2} + n_{2}\right),\\
\displaystyle \dot{P} &= \delta n_{2}I - m_{3}P.
\end{array}
\right.
\end{eqnarray}
onde $H$ é a densidade larval da broca da cana-de-açúcar (\textit{Diatraea saccharalis}), $I$ é a densidade da população de $H$ que é parasitada por $P$, enquanto $P$ é a densidade da larva parasita(\textit{Cotesia flavipes}). Quanto aos parâmetros, tem-se que $r$ representa a taxa de crescimento intrínseco da praga; $K$ é a capacidade de suporte do ambiente; $m_{2}$ e $m_3$ são as taxas de mortalidade de $I$ e $P$, respectivamente; $n_{1}$ é a fração da população da larva que se transforma em pupa em um instante de tempo $t$; $n_{2}$ é a fração da larva parasitada da qual parasitoides adultos emergem em um instante de tempo $t$; $\beta$ é a taxa intrínseca de parasitismo; $\delta$ é o número de parasitoides adultos que emergem de uma larva parasitada em um instante $t$.



%
%	Referencias bibliograficas:
%
\bibliographystyle{acm}  % formatacao do estilo das referencias

%	Bibliografia com o sistema nativo:
%
	% \begin{thebibliography}{9}
	% 	\addbibtocontents
	% 	\bibitem{lamport94}
	% 	  Leslie Lamport,
	% 	  \textit{\LaTeX: a document preparation system},
	% 	  Addison Wesley, Massachusetts,
	% 	  2nd edition,
	% 	  1994.
	% \end{thebibliography}

%	Bibliografia com Bibtex:
%
	\bibliography{./bib/ref}
	% \nocite{lamport94,warnatz06}
	\nocite{warnatz06}

%
%	Apendices:
%
\appendix % formatacao do estilo dos apendices

%	Apendice A - Algoritmos:
	\chapter{Algoritmos}
		\lipsum[29]
		\section{Primeira seção}
			\lipsum[30-31]
		\section{Segunda seção}
			\lipsum[32-33]
		\section{Terceira seção}
			\lipsum[34-35]

\end{document}

\chapter{Introdução}
Teste latexmk
\section{Título da primeira seção}
Texto texto texto texto texto texto texto texto texto texto texto texto texto texto texto texto texto texto texto texto texto.

\subsection{Como inserir citações no texto}

De acordo com \cite{bucur2016nonlocal} texto texto texto texto texto texto texto texto texto texto texto texto texto texto texto texto texto texto texto texto texto texto texto texto texto texto texto.

Texto texto texto  texto texto texto texto texto texto texto texto texto texto texto texto texto texto texto texto texto texto \cite{huang2014numerical,warnatz06}.

\section{Figuras e Tabelas}\label{sec3}

Instruções para a introdução de figuras e tabelas.

%\begin{obs}
%Não esqueçam: tabelas e figuras têm legendas.
%\end{obs}

\subsection{Exemplo: como incluir figuras}
OBS: A legenda deve vir abaixo da figura.

Na Figura \ref{figura01} é possível perceber que texto texto texto texto.
\begin{figure}
\centering
\includegraphics[width=.3\textwidth]{chap/logoppgmap.png}
\caption{ {\small Logo do PPGMAp, UFRGS, Porto Alegre.}}
\label{figura01}
\end{figure}

\subsection{Exemplo: como incluir tabelas}

Na confecção das tabelas é recomendado usar o ambiente \texttt{table}, com legenda acima da tabela e as entradas centralizadas nas colunas, como na Tabela \ref{tabela01}.
\begin{table}[ht] \label{tabela01}
\caption{ \  Metodologia 1- Boole}
\begin{center}
\begin{footnotesize}
\begin{tabular}{ccccccc}
\hline
 $x$  & N=1001  & N=2001  & N=4001  & N=8001
\\ \hline   
0.00  &  0.000467  &  0.000467  &  0.000468    &  0.000468    &  \\ 
10.00  &  0.031540  &  0.031546  &  0.031546    &  0.031546    &  \\
20.00  &  0.612111  &  0.612161  &  0.612163    &  0.612163    & \\ 
30.00  &  12.320687  &  12.318744  &  12.318006    &  12.317711    &  \\ 
40.00  &  1.705255  &  1.705382  &  1.705394    &  1.705398    & \\ 
50.00  &  1.506538  &  1.507221  &  1.507602    &  1.507797    & \\ 
60.00  &  21.900884  &  21.894519  &  21.892193    &  21.891369    &\\ 
70.00  &  1.072764  &  1.072963  &  1.073030    &  1.073067    & \\ 
80.00  &  0.055370  &  0.055387  &  0.055391    &  0.055393    & \\ 
90.00  &  0.002853  &  0.002854  &  0.002854    &  0.002855    & \\ 
100.00  &  0.000042  &  0.000042  &  0.000042    &  0.000042    & \\ 
  \hline
\end{tabular}
\end{footnotesize}
\end{center}
\label{table6P}
\end{table}



\subsection{Exemplo: como incluir equações}

As equações mais importantes são numeradas sequencialmente no texto, com a numeração automaticamente colocada à direita usando o comando \verb!\label{nome-da-equacao}! para identificá-las. A chamada \verb!\eqref{nome-da-equacao}! faz referência à equação, no texto.

EXEMPLO: Considere o sistema de equações \eqref{equm}, dado por:
%\begin{equation}\label{equm}
%\frac{\partial u}{\partial t}=\frac{\partial u}{\partial x}, \quad  %\mathrm{onde} \; u=u(x,t)
%\end{equation}
\begin{eqnarray} \label{equm}
\left\{
\begin{array}{ll}
\vspace{0.1cm}
\displaystyle \dot{H} &= r\left(1-\frac{H}{K}\right) H-n_{1}H-\beta PH, \\
\vspace{0.1cm}
\displaystyle \dot{I} &= \beta HP - I\left(m_{2} + n_{2}\right),\\
\displaystyle \dot{P} &= \delta n_{2}I - m_{3}P.
\end{array}
\right.
\end{eqnarray}
onde $H$ é a densidade larval da broca da cana-de-açúcar (\textit{Diatraea saccharalis}), $I$ é a densidade da população de $H$ que é parasitada por $P$, enquanto $P$ é a densidade da larva parasita(\textit{Cotesia flavipes}). Quanto aos parâmetros, tem-se que $r$ representa a taxa de crescimento intrínseco da praga; $K$ é a capacidade de suporte do ambiente; $m_{2}$ e $m_3$ são as taxas de mortalidade de $I$ e $P$, respectivamente; $n_{1}$ é a fração da população da larva que se transforma em pupa em um instante de tempo $t$; $n_{2}$ é a fração da larva parasitada da qual parasitoides adultos emergem em um instante de tempo $t$; $\beta$ é a taxa intrínseca de parasitismo; $\delta$ é o número de parasitoides adultos que emergem de uma larva parasitada em um instante $t$.
